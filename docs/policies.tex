\PassOptionsToPackage{unicode=true}{hyperref} % options for packages loaded elsewhere
\PassOptionsToPackage{hyphens}{url}
%
\documentclass[]{article}
\usepackage{lmodern}
\usepackage{amssymb,amsmath}
\usepackage{ifxetex,ifluatex}
\usepackage{fixltx2e} % provides \textsubscript
\ifnum 0\ifxetex 1\fi\ifluatex 1\fi=0 % if pdftex
  \usepackage[T1]{fontenc}
  \usepackage[utf8]{inputenc}
  \usepackage{textcomp} % provides euro and other symbols
\else % if luatex or xelatex
  \usepackage{unicode-math}
  \defaultfontfeatures{Ligatures=TeX,Scale=MatchLowercase}
\fi
% use upquote if available, for straight quotes in verbatim environments
\IfFileExists{upquote.sty}{\usepackage{upquote}}{}
% use microtype if available
\IfFileExists{microtype.sty}{%
\usepackage[]{microtype}
\UseMicrotypeSet[protrusion]{basicmath} % disable protrusion for tt fonts
}{}
\IfFileExists{parskip.sty}{%
\usepackage{parskip}
}{% else
\setlength{\parindent}{0pt}
\setlength{\parskip}{6pt plus 2pt minus 1pt}
}
\usepackage{hyperref}
\hypersetup{
            pdftitle={Policies},
            pdfborder={0 0 0},
            breaklinks=true}
\urlstyle{same}  % don't use monospace font for urls
\usepackage[margin=1in]{geometry}
\usepackage{longtable,booktabs}
% Fix footnotes in tables (requires footnote package)
\IfFileExists{footnote.sty}{\usepackage{footnote}\makesavenoteenv{longtable}}{}
\usepackage{graphicx,grffile}
\makeatletter
\def\maxwidth{\ifdim\Gin@nat@width>\linewidth\linewidth\else\Gin@nat@width\fi}
\def\maxheight{\ifdim\Gin@nat@height>\textheight\textheight\else\Gin@nat@height\fi}
\makeatother
% Scale images if necessary, so that they will not overflow the page
% margins by default, and it is still possible to overwrite the defaults
% using explicit options in \includegraphics[width, height, ...]{}
\setkeys{Gin}{width=\maxwidth,height=\maxheight,keepaspectratio}
\setlength{\emergencystretch}{3em}  % prevent overfull lines
\providecommand{\tightlist}{%
  \setlength{\itemsep}{0pt}\setlength{\parskip}{0pt}}
\setcounter{secnumdepth}{0}
% Redefines (sub)paragraphs to behave more like sections
\ifx\paragraph\undefined\else
\let\oldparagraph\paragraph
\renewcommand{\paragraph}[1]{\oldparagraph{#1}\mbox{}}
\fi
\ifx\subparagraph\undefined\else
\let\oldsubparagraph\subparagraph
\renewcommand{\subparagraph}[1]{\oldsubparagraph{#1}\mbox{}}
\fi

% set default figure placement to htbp
\makeatletter
\def\fps@figure{htbp}
\makeatother


\title{Policies}
\author{}
\date{\vspace{-2.5em}}

\begin{document}
\maketitle

dashboard Click the icon to download a PDF copy of the course policies.

\hypertarget{course-learning-objectives}{%
\subsubsection{Course Learning
Objectives}\label{course-learning-objectives}}

\begin{itemize}
\tightlist
\item
  Learn to explore, visualize, and analyze data in a reproducible and
  shareable manner
\item
  Gain experience in data wrangling and munging, exploratory data
  analysis, predictive modeling, and data visualization
\item
  Work on problems and case studies inspired by and based on real-world
  questions and data
\item
  Learn to effectively communicate results through written assignments
  and final project presentation
\end{itemize}

\hypertarget{units}{%
\paragraph{Units}\label{units}}

\begin{itemize}
\tightlist
\item
  Unit 1: Collecting, wrangling, \& visualizing data
\item
  Unit 2: Making rigorous conclusions
\item
  Unit 3: Introductory modeling techniques
\end{itemize}

\hypertarget{course-communicty}{%
\subsubsection{Course Communicty}\label{course-communicty}}

\hypertarget{duke-community-standard}{%
\paragraph{Duke Community Standard}\label{duke-community-standard}}

As a student in this course, you have agreed to uphold the
\href{https://studentaffairs.duke.edu/conduct/about-us/duke-community-standard}{Duke
Community Standard} as well as the practices specific to this course.

\hypertarget{inclusive-community}{%
\paragraph{Inclusive Community}\label{inclusive-community}}

It is my intent that students from all diverse backgrounds and
perspectives be well-served by this course, that students' learning
needs be addressed both in and out of class, and that the diversity that
the students bring to this class be viewed as a resource, strength and
benefit. It is my intent to present materials and activities that are
respectful of diversity and in alignment with
\href{https://provost.duke.edu/initiatives/commitment-to-diversity-and-inclusion}{Duke's
Commitment to Diversity and Inclusion}. Your suggestions are encouraged
and appreciated. Please let me know ways to improve the effectiveness of
the course for you personally, or for other students or student groups.

\hypertarget{accessibility}{%
\paragraph{Accessibility}\label{accessibility}}

If there is any portion of the course that is not accessible to you due
to challenges with technology or the course format, please let me know
so we can make accommodations.

In addition to accessibility issues experienced during the typical
academic year, I recognize that remote learning may present additional
challenges. Students may be experiencing unreliable wi-fi, lack of
access to quiet study spaces, varied time-zones, or additional
responsibilities while studying at home. If you are experiencing these
or other difficulties, please contact me to discuss possible
accommodations.

Duke University is committed to providing equal access to students with
documented disabilities. Students with disabilities may contact the
\href{https://access.duke.edu/students}{Student Disability Access Office
(SDAO)} to ensure your access to this course and to the program.Students
are encouraged to register with the SDAO as soon as they begin the
program. Please note that accommodations are not provided retroactively.
More information can be found online at
\href{https://access.duke.edu/requests}{access.duke.edu}.

\hypertarget{academic-honesty}{%
\paragraph{Academic Honesty}\label{academic-honesty}}

By enrolling in this course, you have agreed to abide by and uphold the
provisions of the
\href{https://studentaffairs.duke.edu/conduct/about-us/duke-community-standard}{Duke
Community Standard} as well as the policies specific to this course. Any
violations will automatically result in a grade of 0 on the assignment
and will be reported to
\href{https://studentaffairs.duke.edu/conduct}{Office of Student
Conduct} for further action.

\begin{itemize}
\tightlist
\item
  You may not discuss or otherwise work with others on the exams.
  Unauthorized collaboration or using unauthorized materials will be
  considered a violation for all students involved. More details will be
  given closer to the exam date.
\item
  \textbf{Reusing code}: Unless explicitly stated otherwise, you may
  make use of online resources (e.g.~StackOverflow) for coding examples
  on assignments. If you directly use code from an outside source (or
  use it as inspiration), you must or explicitly cite where you obtained
  the code. Any recycled code that is discovered and is not explicitly
  cited will be treated as plagiarism.
\item
  On individual assignments, you may not directly share code or write up
  with other students. On team assignments, you may not directly share
  code or write up with another team. Unauthorized sharing of the code
  or write up will be considered a violation for all students involved. 
\end{itemize}

\hypertarget{where-to-find-help}{%
\paragraph{Where to find help}\label{where-to-find-help}}

\begin{itemize}
\tightlist
\item
  If you have a question during lecture or lab, feel free to ask it!
  There are likely other students with the same question, so by asking
  you will create a learning opportunity for everyone.
\item
  The teaching team is here to help you be successful in the course. You
  are encouraged to attend any of the office hours posted on the home
  page to ask questions as your study the course content and work
  through assignments. A lot of questions are most effectively answered
  in-person, so office hours are a valuable resource. Please use them!
\item
  Outside of class and office hours, any general questions about course
  content or assignments should be posted on Piazza since there are
  likely other students with the same questions. The questions you post
  will be visible to the entire class, so please email the instructor
  directly with any specific questions about grades or personal matters.
\end{itemize}

Sometimes you may need help with the class that is beyond what can be
provided by the teaching team. In that instance, I encourage you to
visit the Academic Resource Center.

The \href{https://arc.duke.edu}{Academic Resource Center (ARC)} offers
free services to all students during their undergraduate careers at
Duke. Services include Learning Consultations, Peer Tutoring and Study
Groups, ADHD/LD Coaching, Outreach Workshops, and more. Because learning
is a process unique to every individual, they work with each student to
discover and develop their own academic strategy for success at Duke.
Contact the ARC to schedule an appointment. Undergraduates in any year,
studying any discipline can benefit!

\#\#\#\# Communication

All assignments and course materials may be found on the course website
and \href{https://github.com/sta199-summer2021}{GitHub} page. There is
also an up-to-date \href{./schedule.html}{course schedule} where you can
find the lecture notes discussed in each class meeting, assignment
deadlines, and reading assignments to help you prepare for each class.

Announcements may also be sent to the class by email, so please check
your email regularly.

\hypertarget{activities-assessments}{%
\subsubsection{Activities \& Assessments}\label{activities-assessments}}

The following activities and assessments will help you successfully
achieve the course learning objectives. By experiencing the course
content in different ways, you will not only gain a better understanding
of data science, but you will also get experiences that can guide you as
you apply what you've learned in future academic and professional
projects.

\hypertarget{homework}{%
\paragraph{Homework}\label{homework}}

There will be three individual homework assignments in this course, one
corresponding to each unit. In homework, you will apply what you've
learned during lecture and lab to complete data analysis tasks. You may
discuss homework assignments with other students; however, homework
should be completed and submitted individually. Homework must be typed
up using R Markdown and GitHub and submitted as a PDF in Gradescope.

Individual homework extensions will only be given for extenuating
circumstances. Please contact me directly if you have an extenuating
circumstance that prohibits you from completing the homework by the
stated due date.

\hypertarget{labs}{%
\paragraph{Labs}\label{labs}}

In labs, you will apply the concepts discussed in lecture to various
data analysis scenarios, with a focus on the computation. You will
complete lab assignments individually, but can discuss with your
classmates. You are expected to use your Git repository on the course's
GitHub page as as the central platform for work. Commits to this
repository will be used as a component of the lab grade. Lab assignments
will be completed using R Markdown, correspond to an appropriate GitHub
repository, and submitted as a .pdf file to Gradescope.

\emph{To accommodate unexpected events, the lowest lab grade will be
dropped at the end of the course for all students.}

\hypertarget{exams}{%
\paragraph{Exams}\label{exams}}

The exams are an opportunity to assess the knowledge and skills you've
learned. Both exams will be take-home assignments that you are expected
to complete individually. Each exam will include small analysis and
computational tasks related to the content discussed in lectures,
application exercises, homework assignments, and labs. More details
about the content and structure of the exams will be discussed during
the semester.

\hypertarget{final-project}{%
\paragraph{Final Project}\label{final-project}}

The purpose of the project is to apply what you've learned throughout
the semester to analyze an interesting data-based research question. The
project will be completed in teams, and each team will present their
results during the lecture on Wednesday, July 21. \emph{You must
complete the final project and present your work in class to pass the
course.}

\hypertarget{participation-application-exercises}{%
\paragraph{Participation \& Application
Exercises}\label{participation-application-exercises}}

Application exercises (AEs) give you an opportunity to practice using
the statistical concepts and/or code discussed in lecture on short data
analyses. They will typically be started during class and may be
assigned to be completed by the next class meeting. If so, these AEs are
due by the end of the next day; for instance, an AE associated with a
lecture on Monday will be due Tuesday at 11:59p. AEs will be graded
based on a good-faith effort has been made in attempting all parts.
Successful on-time completion of at least 90\% of AE will result in full
points for that AE; anything lower than that will be assigned points
accordingly. In general, these assignments are shorter than homework
assignments. \emph{To accommodate unexpected events, the two lowest
Application Exercise grades will be dropped at the end of the course for
all students.}

\hypertarget{data-visualization-examples}{%
\paragraph{Data Visualization
Examples}\label{data-visualization-examples}}

Everyone will sign up to share a data visualization that they find
meaningful and believe does a good job at presenting a story. These will
be accompanied by a short written submission. The goal of this
activities is for you to think about how the material you're learning in
the course can connect with your experiences and society at large. More
details will be provided before the first presentation.

\hypertarget{grade-calculation}{%
\paragraph{Grade Calculation}\label{grade-calculation}}

The final grade will be calculated as follows:

\begin{longtable}[]{@{}ll@{}}
\toprule
{} & {}\tabularnewline
\midrule
\endhead
Homework & 17.5\%\tabularnewline
Labs & 15\%\tabularnewline
Exam 1 & 17.5\%\tabularnewline
Exam 2 & 17.5\%\tabularnewline
Final Project & 17.5\%\tabularnewline
Participation \& Application Exercises & 10\%\tabularnewline
Data Visualization Examples & 5\%\tabularnewline
\bottomrule
\end{longtable}

Class attendance in lecture and lab is a firm expectation; frequent
absences or tardiness will be considered a legitimate cause for grade
reduction.

If you have a cumulative numerical average of 90 - 100, you are
guaranteed at least an A-, 80 - 89 at least a B-, and 70 - 79 at least a
C-. The exact ranges for letter grades will be determined after Exam 2.

Students are required to enroll in both Summer Sessions I and II in
order to obtain credit for the course.

\hypertarget{regrade-requests}{%
\subsubsection{Regrade Requests}\label{regrade-requests}}

Regrade requests should be submitted through the regrade request from on
Gradescope. \textbf{Requests for a regrade must be made within a week of
when the assignment is returned.} Due to the time consuming nature of
regrades, requests submitted later will not be regraded. Requests will
be honored if there is an error in the grade calculation or a correct
answer was mistakenly marked as incorrect. Please note that by
submitting a regrade request, your entire assignment may be regraded and
you may potentially lose points. Therefore, you should attend office
hours to ask a member of the teaching team about your grading feedback
before submitting a regrade request.

No grades will be changed after the final project presentations.

\emph{Note: Grades can only be changed by the instructor. Teaching
Assistants cannot change grades on returned assignments.}

\hypertarget{make-up-policy}{%
\subsubsection{Make-up Policy}\label{make-up-policy}}

Students who miss a class due to a scheduled varsity trip, religious
holiday, or short-term illness should fill out an online
\href{https://trinity.duke.edu/undergraduate/academic-policies/athletic-varsity-participation}{NOVAP},
\href{https://trinity.duke.edu/undergraduate/academic-policies/religious-holidays}{Religious
Observance Notification}, or
\href{https://trinity.duke.edu/undergraduate/academic-policies/illness}{Incapacitation
Form}, respectively. These excused absences do not excuse you from
assigned homework. It will still be your responsibility to submit
relevant assignments in accordance with the deadline.

If you have a personal or family emergency or health condition that
affects your ability to participate in class, you should contact your
academic dean's office. More information about this procedure may be
found on the
\href{https://trinity.duke.edu/undergraduate/academic-policies/personal-emergencies}{Personal
Emergencies page} or provided by your academic dean.

\textbf{Exam dates cannot be changed and no make-up exams will be
given.} If you must miss an exam, your absence must be officially
excused before the exam due date. If your absence is excused, the
missing exam grade will be imputed at the end of the semester based on
your performance on other relevant course assignments.

The final project presentations will be held during lecture on
Wednesday-Thursday July 21-22. \textbf{You must complete the final
project and present your work during this period in order to pass the
course.}

\hypertarget{late-work}{%
\subsubsection{Late Work}\label{late-work}}

To accomodate different time zones, homework or lab assignments
submitted late but within 12 hours of the deadline may be accepted with
no penalty. After the 12 hours, there is a 20\% penalty for each day the
assignment is late.

If there are extenuating circumstances that prevent you from completing
an assignment by the stated due date, please let me know as soon as
possible.

\textbf{Late work will not be accepted for the exams or the final
project.}

\end{document}
